\chapter{基于十亿亿次神威超算的高分辨率唐山大地震模拟}
\label{chap:earthquake}

\section{背景知识和算法概述}

在中国传统文化中,对学者的最高评价是“上知天文,下知地理”。自然科学与技术在上世纪取得了长足的进步,然而,地球内部的构造、运动模式、可预测性等问题对于科学家来说仍旧充满未知与神秘。因此,地震等对人类社会生命财产安全造成巨大损失的重大灾害则成为了等待着科学家破译的最终科学挑战之一。

公元132年,东汉著名天文学家张衡设计的地动仪可能是中国历史上最早的抗震减灾科学工作之一(如图所以)。地动仪有八个方位,每个方位上均有口含龙珠的龙头,在每条龙头的下方都有一只蟾蜍与其对应。任何一方如有地震发生,该方向龙口所含龙珠即落入蟾蜍口中,由此便可测出发生地震的方向\cite{}。

这个约在两千年前发明的地动仪中,


有八条龙在他们的嘴里叼着球,能够在千里之外的地震中发现,并展现出与现代地震相似的技术特征 测量仪器起源于1880年代

张衡所设计的地震仪, 如图所示,citep {milne1886earthquakes}。

\section{地震模拟软件框架}


%!TEX root = main.tex
\begin{resume}

  \resumeitem{个人简历}

  1990 年 9 月 13 日出生于广东省兴宁县。

  2009 年 9 月考入中山大学软工学院软件工程专业,2013 年 7 月本科毕业并获得工学学士学位。

  2013 年 9 月免试进入清华大学计算机科学与技术系攻读博士学位至今。

  博士在读期间,2016 年 6 月至 2016 年 9 月在斯坦福大学(Stanford University)地球物理系做访问研究;2016 年 11 月至 2017 年 5 月在伦敦帝国理工学院(Imperial College London)计算机系做访问研究。

  \researchitem{发表的学术论文} % 发表的和录用的合在一起

  % 1. 已经刊载的学术论文(本人是第一作者,或者导师为第一作者本人是第二作者)
  \begin{publications}
    \item Haohuan Fu, \textbf{Conghui He}, Bingwei Chen et al. ``18.9-Pflops Nonlinear Earthquake Simulation on Sunway TaihuLight : Enabling Depiction of 18-Hz and 8-Meter Scenarios." In High Performance Computing, Networking, Storage and Analysis, SC17, 2017. (CCF 推荐 A 类会议,获得2017年“戈登贝尔”奖,本人为通信作者之一)

    \item \textbf{Conghui He}, Haohuan Fu, Ce Guo, Wayne Luk, and Guangwen Yang. ``A Fully-Pipelined Hardware Design for Gaussian Mixture Models." IEEE Transactions on Computers, 2017. (CCF 推荐 A 类期刊)

    \item Haohuan Fu, \textbf{Conghui He}, Huabin Ruan, Itay Greenspon, Wayne Luk, Yongkang Zheng, Junfeng Liao, Qing Zhang, and Guangwen Yang. ``Accelerating Financial Market Server through Hybrid List Design (abstract only)." In Proceedings of the 2017 ACM/SIGDA International Symposium on Field-Programmable Gate Arrays, pp. 289-290. (CCF 推荐 B 类会议)

    \item \textbf{Conghui He}, Haohuan Fu, Wayne Luk, Weijia Li, and Guangwen Yang. ``Exploring the Potential of Reconfigurable Platforms for Order Book Update." In IEEE International Conference on Field-Programmable Logic and Applications (FPL), 2017. (CCF 推荐 C 类会议)

    \item Haohuan Fu, \textbf{Conghui He}, Wayne Luk, Weijia Li, and Guangwen Yang. ``A Nanosecond-level Hybrid Table Design for Financial Market Data Generators." The 25th IEEE International Symposium on Field-Programmable Custom Computing Machines, 2017. (CCF 推荐 C 类会议)

   
  \end{publications}

  % 2. 尚未刊载,但已经接到正式录用函的学术论文(本人为第一作者,或者
  %    导师为第一作者本人是第二作者)。
  % \begin{publications}[before=\publicationskip,after=\publicationskip]
  %   \item Yang Y, Ren T L, Zhu Y P, et al. PMUTs for handwriting recognition. In
  %     press. (已被 Integrated Ferroelectrics 录用. SCI 源刊.)
  % \end{publications}

  % 3. 其他学术论文。可列出除上述两种情况以外的其他学术论文,但必须是
  %    已经刊载或者收到正式录用函的论文。
  \begin{publications}
        \item Weijia Li, \textbf{Conghui He}, Wayne Luk and Haohuan Fu. ``An FPGA-based tree crown detection approach for remote sensing images." The IEEE International Conference on Field-Programmable Technology, 2017. (CCF 推荐 C 类会议)

 \item \textbf{Conghui He}, Haohuan Fu, Yi Shen, Robert Clapp, and Guangwen Yang. ``Ensemble Full Wave Inversion with Source Encoding." In 79th EAGE Conference and Exhibition 2017. 

    \item \textbf{Conghui He}, Yushu Chen, Haohuan Fu, and Guangwen Yang. Ensemble Full Wave Inversion with Source Encoding. In 77th EAGE Conference and Exhibition 2015.

    \item \textbf{Conghui He}, Haohuan Fu, Bangtian Liu, Huabin Ruan, Guangwen Yang, Hui Yang, and Are Osen. ``A GPU-based Parallel Beam Migration Design." In 2015 SEG Annual Meeting. Society of Exploration Geophysicists, 2015.

    \item Bingwei Chen, \textbf{Conghui He}, Yushu Chen, Haohuan Fu. ``Full Wave Inversion Based on EnKF and Source Encoding" In 2016 SEG Annual Meeting. Society of Exploration Geophysicists.

    \item Haohuan Fu, Junfeng Liao, Wei Xue, Lanning Wang, Dexun Chen, Long Gu, Jinxiu Xu, Nan Ding, Xinliang Wang, \textbf{Conghui He}, Shizhen Xu, et al. ``Refactoring and optimizing the community atmosphere model (CAM) on the sunway taihulight supercomputer." Proceedings of the International Conference for High Performance Computing, Networking, Storage and Analysis. IEEE Press, 2016. (CCF 推荐 A 类会议)

    \item Yushu Chen, Guangwen Yang, Xiao Ma, \textbf{Conghui He}, and Guojie Song. ``A time-space domain stereo finite difference method for 3D scalar wave propagation." Computers \& Geosciences 96 (2016): 218-235. (SCI检索)

    \item Nicholas Clinton, Le Yu, Haohuan Fu, \textbf{Conghui He}, and Peng Gong. ``Global-Scale Associations of Vegetation Phenology with Rainfall and Temperature at a High Spatio-Temporal Resolution." Remote Sensing 6, no. 8 (2014): 7320-7338. (SCI检索)
  \end{publications}

  % \researchitem{研究成果} % 有就写,没有就删除
  % \begin{achievements}
  %   \item 任天令, 杨轶, 朱一平, 等. 硅基铁电微声学传感器畴极化区域控制和电极连接的
  %     方法: 中国, CN1602118A. (中国专利公开号)
  %   \item Ren T L, Yang Y, Zhu Y P, et al. Piezoelectric micro acoustic sensor
  %     based on ferroelectric materials: USA, No.11/215, 102. (美国发明专利申请号)
  % \end{achievements}

\end{resume}

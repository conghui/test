% 如果使用声明扫描页,将可选参数指定为扫描后的 PDF 文件名,例如:
% \begin{acknowledgement}[scan-statement.pdf]
\begin{acknowledgement}

衷心感谢我的导师付昊桓老师一直以来对我的栽培,付老师不仅在学术上给我指导、答疑解惑、修改论文、协调资源,他的人格魅力以及待人处事的风格更是深深影响着我。

特别感谢杨广文老师

衷心感谢我的导师付昊桓老师和杨广文老师多年来对我的悉心照顾。付老师和杨老师在学术与科研中不断给予我帮助,鼓励与启发,一次次为我答疑解惑、修改论文、协调资源,并在科研方向与具体规划上提供了许多建设性意见。

毕业论文的顺利完成,离不开所有帮助过我的老师与同学。
感谢来自清华的黄小猛老师、薛巍老师和刘利老师

;感谢伦敦帝国理工学院的Wayne Luk教授、牛昕宇
与Oskar Mencer研究员,
和
他们一直以来都在关心我的学术与科研,
每一次与他们交流与探讨,我都受益匪浅。
感谢中科院软件所的杨超研究员,伦敦帝国理工学院的Wayne Luk教授与Oskar Mencer研究员,
以及斯坦福大学的Robert Clapp研究员与Biondo Biondi教授,
他们分别从大气模式数值模拟方法、可重构平台与优化方法、
以及地球物理勘探等不同领域为我提供指导,并在论文撰写过程中为我提供了许多宝贵意见。
还要感谢Wayne Luk教授与Robert Clapp研究员为我提供了访学交流的机会。
感谢清华实验室的姜进磊老师,王小鸽老师,武永卫老师,郑纬民老师等多年来给予的关心与照顾。
感谢所有一起并肩奋斗的师兄弟,阮华斌、张诚、李锐喆、
赵文来、胡勇、王英侨、魏腾鹏、陈宇澍、薛志辉、廖俊锋、吕子鈜、徐世真、郑伟杰、靳梦瑶、何聪辉、刘邦天、徐敬蘅、刘加贺、方佳瑞、李维嘉、陈炳伟、张赫等。
感谢Maxeler公司、英伟达北京研发中心以及英特尔公司为本工作提供了最新硬件平台与优化建议。

感谢我的家人与朋友,特别是我的父母,他们一直都是我人生道路中的坚强后盾,也是我不断探知求索的最大动力源泉。

最后,感谢自然科学基金、国家863计划、清华大学博士生短期出国访学基金、清华大学信息科学技术学院登峰基金、帝国理工学院计算机系,斯坦福大学地球物理系等对本人工作的资助。
感谢答辩委员会专家及匿名评审专家的宝贵时间和指导。
\end{acknowledgement}

%!TEX root = main.tex
\thusetup{
  %******************************
  % 注意:
  %   1. 配置里面不要出现空行
  %   2. 不需要的配置信息可以删除
  %******************************
  %
  %=====
  % 秘级
  %=====
  secretlevel={秘密},
  secretyear={10},
  %
  %=========
  % 中文信息
  %=========
  ctitle={面向神威·太湖之光的地震模拟并行优化方法研究},
  cdegree={工学博士},
  cdepartment={计算机科学与技术系},
  cmajor={计算机科学与技术},
  cauthor={何聪辉},
  csupervisor={付昊桓副教授},
  %cassosupervisor={陈文光教授}, % 副指导老师
  %ccosupervisor={某某某教授}, % 联合指导老师
  % 日期自动使用当前时间,若需指定按如下方式修改:
  %cdate={2018年03月},
  %
  % 博士后专有部分
  % cfirstdiscipline={计算机科学与技术},
  % cseconddiscipline={系统结构},
  % postdoctordate={2009年7月——2011年7月},
  % id={编号}, % 可以留空: id={},
  % udc={UDC}, % 可以留空
  % catalognumber={分类号}, % 可以留空
  %
  %=========
  % 英文信息
  %=========
  etitle={Accelerating the Seismic Simulations on Sunway TaihuLight Supercomputer},
  % 这块比较复杂,需要分情况讨论:
  % 1. 学术型硕士
  %    edegree:必须为Master of Arts或Master of Science(注意大小写)
  %             “哲学、文学、历史学、法学、教育学、艺术学门类,公共管理学科
  %              填写Master of Arts,其它填写Master of Science”
  %    emajor:“获得一级学科授权的学科填写一级学科名称,其它填写二级学科名称”
  % 2. 专业型硕士
  %    edegree:“填写专业学位英文名称全称”
  %    emajor:“工程硕士填写工程领域,其它专业学位不填写此项”
  % 3. 学术型博士
  %    edegree:Doctor of Philosophy(注意大小写)
  %    emajor:“获得一级学科授权的学科填写一级学科名称,其它填写二级学科名称”
  % 4. 专业型博士
  %    edegree:“填写专业学位英文名称全称”
  %    emajor:不填写此项
  edegree={Doctor of Philosophy},
  emajor={Computer Science and Technology},
  eauthor={Conghui He},
  esupervisor={Associate Professor Haohuan Fu},
  %eassosupervisor={Chen Wenguang},
  % 日期自动生成,若需指定按如下方式修改:
  %edate={March, 2018}
  %
  % 关键词用“英文逗号”分割
  %ckeywords={天然地震, 地震勘探, 高性能计算, 算法, 并行优化, 全波形反演, 正演, 体系结构, 加速},
  %ekeywords={\TeX, \LaTeX, CJK, template, thesis}
}

% 定义中英文摘要和关键字
\begin{cabstract}
天然大地震的破坏范围广、突发性强、次生灾害严重,不仅剧烈地冲击着国家的社会生活和经济活动,也对人们的心理造成了重大的影响。水能覆舟亦能载舟,局部的人工地震却是勘探石油与天然气等化石能源的重要手段。地震模拟研究不仅可以深入理解地球内部结构和地震演变机制、定量评估地震风险,还能有助于油气资源的开采,保证国家能源供应和维持工业生产正常运行。“神威·太湖之光”超级计算机采用我国自主研制的申威26010片上异构众核处理器,是世界上首台峰值性能超过十亿亿次的超级计算机,在性能、规模和能效上均展现出独特的优势。地震模拟作为超级计算机的主要服务对象,一直以来都是高性能计算的主要挑战之一。近年来随着超级计算机的规模不断扩大、架构持续更新,科学家对模拟精度、时间和空间分辨率的需求也不断提升,面向“神威·太湖之光”超级计算机的地震模拟研究也面临着前所未有的巨大挑战。
%“神威·太湖之光”为各类科学应用提供了强大的算力支撑


本文以“神威·太湖之光”超级计算机为目标开发和优化平台,研究大规模地震模拟和地震资料处理在神威超算上的并行优化方法,分别从地震正演和反演算法、申威26010处理器架构、神威超算大规模并行等角度出发提出了一系列优化方法,将地震模拟的规模、性能、分辨率和精度都推上了一个新的台阶。主要贡献包括:
  \begin{itemize}
    \item 提出分时动态区域变分辨率正演方法和集合全波形反演方法,从算法层面优化正演效率,并作为反演的基础。集合全波形反演方法同时提升了反演效率和反演准确率。
    
    \item 提出了面向申威异构众核处理器的地震正演并行优化方法,分别从降低内存传输总量和提升内存传输带宽两方面打破申威处理器的内存墙。然后提出实时压缩方案,在可接受的精度损失下支持更大规模的地震模拟,将应用程序的可用内存大小和带宽提升到全新高度。
    
    \item 提出了面向神威超算的大规模并行优化方法,将地震模拟从单核心高效地扩展到神威超算上千万核心,并提出了针对大规模并行的通信和IO优化方案。

    \item 仿真了非线性唐山大地震模拟和石油勘探全波形反演成像。非线性唐山大地震模拟使用神威超算千万核心取得了18.9 Pflops峰值性能。石油勘探反演成像成功地反演了Marmousi模型,展示了集合全波形反演算法在性能和准确率上的优势。
  \end{itemize}

\end{cabstract}

% 如果习惯关键字跟在摘要文字后面,可以用直接命令来设置,如下:
 \ckeywords{地震模拟, 高性能计算, 神威·太湖之光, 并行优化, 算法设计}

\begin{eabstract}
  While science has made significant progresses in various domains in the current and last century, the interior of the earth still remains largely unknown to the scientists. As a result, the earthquakes, which are significant disasters leading to huge losses in various aspects, are still among the ultimate scientific challenges that wait for deciphering of their development mechanisms and formation of scientific prediction capabilities. Moreover, the lack of understanding of the earth makes it challenging in finding the oil and gas from the undergrounds especially in complex areas.  Thus, the capability of simulating the seismic wave propagation in an accurate way is a key factor along the process of resolving the mysteries of earthquakes and imaging the undergrounds. 

  The "Sunway TaihuLight" supercomputer adopts China's homegrown SW2610 on-chip heterogeneous processors, and exhibits unique advantages in terms of performance, scale, interconnection and energy efficiency. In recent years, the scope, resolution, and data volume of seismic simulation have been continuously increasing. Accelerating the seismic simulations on China's homegrown heterogeneous many-core architecture faces severe challenges such as computation, storage, bandwidth, communication, and IO. 

  This thesis reports our efforts on porting and redesiging the seismic applications on the world's No.1 supercomputer, the Sunway TaihuLight, which achieves significantly improved simulation capabilities with the increased computing power of 125 Pflops of this new system. The main contributions are as follows:

\begin{itemize}
  \item a high performance design of accelerating the wavefield propagation in 3D elastic media by approximating the constant Q propagation as well as the ensemble full wave inversion algorithm with source encoding (EnFWI). 

  \item an elaborate memory scheme that breaks the memory wall on SW26010 as well as an on-the-fly compression scheme that pushes the application-available memory size and bandwidth to the next level. The compression scheme expands both the highest performance and the largest problem size we can achieve on Sunway TaihuLight.

  \item a customized parallelization scheme that employs the 10 million cores efficiently at both the process level and the thread level as well as a complete set of solutions for communication and IO challenges.

  \item two use cases including a seismic simulation framework on Sunway TaihuLight that simulates the Tangshan earthquake in a massively parallel way, and an earth imaging scenario that inverts the Marmousi model using the EnFWI method we proposed. 

\end{itemize}

\end{eabstract}

\ekeywords{Seismic Applications, High Performance Optimization, Sunway TaihuLight, Algorithm Design}

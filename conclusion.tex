%!TEX root = main.tex
\chapter{总结与展望} % (fold)
\label{ch:总结与展望}

\section{论文工作总结}

地震模拟和地震资料处理对高性能计算的需求是无止境,在进一步提升地震模拟计算性能,提高模拟范围、时空分辨率、模拟精度等方面都对高性能计算提出了重大挑战。本文工作以目前世界最快的神威太湖之光超级计算机为目标开发和优化平台,研究大规模地震模拟和地震资料处理在神威超算上的并行优化方法。本文研究分别从地震正演和反演算法、申威26010处理器架构、神威超算大规模并行等角度出发提出了一系列优化方法,将地震模拟的规模、性能、分辨率和精度都推上了一个新的台阶。

高性能应用优化是交叉学科研究,若不了解目标科学应用场景和具体算法流程,仅在代码层面进行优化很容易受到原有程序算法和逻辑的制约,优化效果非常有限。本文研究首先在在算法层面提出了分时动态区域变分辨率正演方法和集合全波形反演方法。分时动态区域正演算法利用地震波传播初期未填充模拟区域的现象,通过分时确定最大传播范围,忽略波前未到达的区域,从而提升计算效率。分时变分辨率正演方法利用地震波能量在地球内部不断衰减的特性,结合稳定性和频散条件,分时动态改变波场时空分辨率,并从二维声波方程扩展到三维弹性波方程。分时动态区域正演和分时变分辨率正演方法可结合使用,形成分时动态变分辨率正演方法,够保证正演精度的同时获得了可观的性能提升。然后本文在全波形反演方法的基础上提出基于震源编码的集合全波形反演方法。集合全波形反演方法通过集合卡尔曼滤波的集合协方差来近似完全反演中协方差算子,然后引入了震源编码技术将所有的震源和地震记录分别进行叠加,在扩大全波形反演收敛域和降低噪音敏感度的同时极大了提升了计算效率。

随后,本文研究在体系结构层面提出了面向申威异构众核处理器的地震正演并行优化方法。分别从降低内存传输总量和提升内存传输带宽来突破申威处理器的内存墙。本文推导出最小DMA数据传输方案完成有限差分运算,并借助从核寄存器通信更新有限差分边界。在增大DMA数据传输带宽方面,本文提出了共位数组融合和数据布局转换两种方法对底层数据结构进行改动,能够显著提升DMA数据传输带宽。在内存容量达到极限的情况下,本文提出了实时压缩/解压缩方案对模拟变量进行压缩,在可接受的精度损失情况下支持更大规模的地震模拟,将应用程序的可用内存大小和带宽提升到全新的高度。

接着,本文提出了把大规模科学应用扩展到神威超算全机时遇到的挑战和相应的解决方法,分别从三个方面进行描述:多层级并行任务划分、大规模通信优化和IO优化。多层级并行任务划分方案灵活根据不同计算单元的特性,将完整的计算区域和计算任务分配到不同的计算单元,最大化不同计算单元的性能,并以接近线性的效率将地震模拟从单核心和单核组扩展到神威超算上千万核心。大规模通信优化介绍了两种通信优化方法,通过计算通信重叠隐藏通信时间和使用虚拟网格降低通信量。大规模并行中IO也是瓶颈之一,本文介绍了神威超算下的大规模IO优化方法,用多进程串行IO方式替代MPI-IO方式,然后对进程进行IO分组和平衡IO,最后结合地震数据的特征使用LZ4压缩方案降低输出文件大小,节约IO带宽和存储空间

最后,本文以真实的地震模拟算例为背景,应用前文所述的地震模拟并行优化方法,展现了唐山大地震和石油物探全波形反演算法在神威太湖之光上取得的性能提升以及模拟结果。非线性唐山大地震模拟使用了神威超算千万核心,模拟范围达到了$320km\times 320km \times 40km$,分辨率高达$8m$,频率高达18Hz,峰值性能达到18.9 PFlops,这是迄今为止最大规模、最高分辨率的地震模拟,且获得了2017年戈登贝尔奖。随后,本研究在高效正演的基础上使用了集合全波形反演算法反演了石油物探的Marmousi模型,借助神威超算系统的各项资源优势,取得了良好的性能结果。反演结果显示,集合全波形反演算法与传统全波形反演方法和基于震源编码的全波形反演方法相比具有更大的收敛域和更低的噪音敏感度。

综上所述,从地震正演和反演算法、单节点申威26010处理器和神威太湖之光超算的大规模并行三个层面分别进行优化是大规模地震模拟并行优化研究高效的优化策略。这种优化策略不仅适用于地震模拟,对其他类$Stencil$运算的科学应用也具有一定的普适性。

\section{未来展望}

地震模拟对高性能计算的需求是永无止境的,正演算法作为地震模拟和反演的核心,其优化也是永无止境的。未来可结合神威超算的特殊架构探索更适合神威超算的正演算法,并对算法进行更深入的优化。集合全波形反演方法也可在更多的模型如SEG盐丘模型、BP模型上进行实验验证。

在结构优化层面可以进一步探索面向申威26010处理器的并行优化。当前工作研究的是星型$Stencil$运算在申威26010处理器上的优化,未来可以研究更复杂的$Stencil$优化,并设计更巧妙的寄存器通信策略进一步打破申威处理器的内存墙。在大规模并行优化层面可建立统一的进程间和线程间通信开销模型并设计出最优的通信方案。

实时压缩方案不仅适用于地震模拟,也适用于其他访存受限的科学应用。目前的压缩算法较为朴素,限制将32位浮点数压缩为16位浮点数。未来可设计更多的压缩算法,并改变压缩率,将32位浮点数压缩为24位或者8位浮点数。

最后,大规模地震模拟的普适性也可进一步提升。本文研究的唐山大地震模拟框架适用于简单平原地形,未来的地震模拟工作可研究复杂地形的地震模拟,如汶川地震模拟。此外,地震模型和其他模型如地面建筑、水力、交通等模型相耦合,模拟地震发生后地面建筑、水力电力和交通的破坏情况,可为抗震救灾提供预案。

% chapter 总结与展望 (end)